\begin{center}
  \large
  \textbf{PEMANFAATAN ALGORITMA GENETIKA UNTUK PROSES PENJADWALAN PERKULIAHAN DI DEPARTEMEN TEKNIK KOMPUTER ITS}
\end{center}
\addcontentsline{toc}{chapter}{ABSTRAK}
% Menyembunyikan nomor halaman
\thispagestyle{empty}

\begin{flushleft}
  \setlength{\tabcolsep}{0pt}
  \bfseries
  \begin{tabular}{ll@{\hspace{6pt}}l}
  Nama Mahasiswa / NRP&:& Muhammad Zakariya Nur Ramdhani / 07211940000016\\
  Departemen&:& Teknik Komputer FTEIC - ITS\\
  Dosen Pembimbing&:& 1. Dr. Diah Puspito Wuklandari, S.T., M.Sc.\\
  & & 2. Dr. Supeno Mardi Susiki Nugroho, ST., MT.\\
  \end{tabular}
  \vspace{4ex}
\end{flushleft}
\textbf{Abstrak}

% Isi Abstrak
Penjadwalan mata kuliah merupakan kegiatan yang sangat krusial bagi 
terselenggaranya kegiatan perkuliahan yang baik bagi sebuah jurusan di 
universitas atau perguruan tinggi. Sebuah penjadwalan dikatakan baik jika 
hasil penjadwalan tersebut dapat dilaksanakan tidak hanya bagi dosen yang mengajar, 
tetapi juga dapat dilaksanakan oleh para mahasiswa yang mengambil mata kuliah tersebut. 
Proses penyusunan penjadwalan mata kuliah di Departemen Teknik Komputer ITS saat ini masih 
dilakukan secara konvensional. Proses penjadwalan konvensional ini bisa memakan waktu yang 
lama dari proses rapat hingga jadwal selesai. Kendala ketersediaan dosen, jumlah mata kuliah, 
jumlah ruangan dan jumlah mahasiswa menjadi tantangan dalam proses penjadwalan karena harus 
dipertimbangkan agar tidak terjadi bentrok dalam hasil penjadwalan. Masalah-masalah yang ada 
dalam proses penjadwalan mata kuliah ini bisa diminimalisir dengan menggunakan teknologi 
yang ada sehingga dihasilkan proses penjadwalan yang optimal sesuai dengan batasan-batasan yang ditentukan.
Salah satu metode yang dapat digunakan dalam mengatasi masalah penjadwalan adalah dengan memanfaatkan 
metode Algoritma Genetika. Algoritma Genetika merupakan teknik untuk mencari penyelesaian optimal dari 
sebuah permasalahan yang memiliki banyak solusi. Teknik ini akan mencari penyelesaian dari beberapa 
solusi yang ada sampai diperoleh penyelesaian terbaik sesuai dengan kriteria yang telah ditentukan sebelumnya. 
Kriteria-kriteria ini biasa dikenal dengan fitness. Oleh karena itu, pada penelitian ini akan memanfaatkan Algoritma 
Genetika untuk proses penjadwalan di Departemen Teknik Komputer ITS.

\vspace{2ex}
\noindent
\textbf{Kata Kunci: \emph{Penjadwalan, Algoritma, Genetika}}