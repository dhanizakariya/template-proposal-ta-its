\begin{flushleft}
  % Ubah kalimat berikut sesuai dengan nama departemen dan fakultas
  \textbf{Departemen Teknik Komputer - FTEIC}\\
  \textbf{Institut Teknologi Sepuluh Nopember}\\
\end{flushleft}

\begin{center}
  % Ubah detail mata kuliah berikut sesuai dengan yang ditentukan oleh departemen
  \underline{\textbf{EC224701 - PROPOSAL TUGAS AKHIR - 2 SKS}}
\end{center}

\begin{adjustwidth}{-0.2cm}{}
  \begin{tabular}{lcp{0.7\linewidth}}

    % Ubah kalimat-kalimat berikut sesuai dengan nama dan NRP mahasiswa
    Nama Mahasiswa &:& Muhammad Zakariya Nur Ramdhani \\
    Nomor Pokok &:& 0721 19 4000 0016 \\

    % Ubah kalimat berikut sesuai dengan semester pengajuan proposal
    Semester &:& Ganjil 2022/2023 \\

    % Ubah kalimat-kalimat berikut sesuai dengan nama-nama dosen pembimbing
    Dosen Pembimbing &:& 1. Dr.\ Diah Puspito Wulandari, S.T., M.Sc \\
    & & 2. Dr.\ Supeno Mardi Susiki Nugroho, ST., MT. \\

    % Ubah kalimat berikut sesuai dengan judul tugas akhir
    Judul Tugas Akhir &:& \textbf{Pemanfaatan Algoritma Genetika untuk Proses Penjadwalan Perkuliahan di Departemen Teknik Komputer ITS} \\
    %& & \textbf{Anti Gravitasi} \\

    Uraian Tugas Akhir &:& \\
  \end{tabular}
\end{adjustwidth}

% Ubah paragraf berikut sesuai dengan uraian dari tugas akhir
Penjadwalan mata kuliah merupakan kegiatan yang sangat krusial bagi 
terselenggaranya kegiatan perkuliahan yang baik bagi sebuah jurusan di 
universitas atau perguruan tinggi. \linebreak Penjadwalan dikatakan baik jika jadwal yang dihasilkan dapat 
dilaksanakan baik olek dosen pengajar maupun mahasiswa. Proses penjadwalan perkuliahan 
di Departemen Teknik Komputer ITS saat ini masih dilakukan secara konvensional. 
Kendala ketersediaan dosen, jumlah mata kuliah, jumlah ruangan dan jumlah mahasiswa menjadi tantangan dalam proses penjadwalan karena harus 
dipertimbangkan agar dapat dihasilkan jadwal yang baik. Banyaknya variabel yang harus dipertimbangkan, 
menyebabakan penjadwalan secara konvensional cenderung lama. Oleh karena itu, pada penelitian ini 
akan memanfaatkan Algoritma Genetika untuk proses penjadwalan di Departemen Teknik Komputer ITS. 
Algoritma Genetika merupakan teknik untuk mencari penyelesaian optimal dari 
sebuah permasalahan yang memiliki banyak solusi. Teknik ini akan mencari penyelesaian dari beberapa 
solusi yang ada sampai diperoleh penyelesaian terbaik sesuai dengan kriteria yang telah ditentukan sebelumnya.

\vspace{1ex}

\begin{flushright}
  % Ubah kalimat berikut sesuai dengan tempat, bulan, dan tahun penulisan
  Surabaya, 14 Februari 2023
\end{flushright}
\vspace{1ex}

\begin{center}

  \begin{multicols}{2}

    Dosen Pembimbing 1
    \vspace{12ex}

    % Ubah kalimat-kalimat berikut sesuai dengan nama dan NIP dosen pembimbing pertama
    \underline{Dr.\ Diah Puspito Wulandari, S.T., M.Sc} \\
    NIP.\ 19801219 200501 2 000

    \columnbreak

    Dosen Pembimbing 2
    \vspace{12ex}

    % Ubah kalimat-kalimat berikut sesuai dengan nama dan NIP dosen pembimbing kedua
    \underline{Dr.\ Supeno Mardi Susiki Nugroho, ST., MT.} \\
    NIP. 19700313 199512 1 001

  \end{multicols}
  \vspace{6ex}

  Mengetahui, \\
  % Ubah kalimat berikut sesuai dengan jabatan kepala departemen
  Kepala Departemen Teknik Komputer FTEIC - ITS
  \vspace{12ex}

  % Ubah kalimat-kalimat berikut sesuai dengan nama dan NIP kepala departemen
  \underline{Dr.\ Supeno Mardi Susiki Nugroho, ST., MT.} \\
  NIP. 19700313 199512 1 001

\end{center}
