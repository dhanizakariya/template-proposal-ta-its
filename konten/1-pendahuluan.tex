\chapter{PENDAHULUAN}

\section{Latar Belakang}

% Ubah paragraf-paragraf berikut sesuai dengan latar belakang dari tugas akhir
Penjadwalan mata kuliah merupakan kegiatan yang sangat krusial bagi terselenggaranya kegiatan perkuliahan yang baik bagi sebuah jurusan di universitas atau perguruan tinggi. 
Sebuah penjadwalan dikatakan baik jika hasil penjadwalan tersebut dapat dilaksanakan tidak hanya bagi dosen yang mengajar, tetapi juga dapat dilaksanakan oleh para mahasiswa yang mengambil mata kuliah tersebut(  \cite{ramadhani2021perancangan}).
Penjadwalan mata kuliah dapat diartikan sebagai proses pengalokasian kegiatan perkuliahan yang terdiri atas dosen pengampu mata kuliah, ruang kuliah, mahasiswa dan jadwal pelaksanaan yang dilaksanakan dalam waktu satu minggu dan rentang waktu satu hari(\cite{Mone2021}).
Proses penyusunan penjadwalan mata kuliah di Departemen Teknik Komputer ITS saat ini masih dilakukan secara konvensional. Proses penjadwalan konvensional ini bisa memakan waktu yang lama hingga 1 minggu dari proses rapat hingga jadwal selesai. 
Kendala ketersediaan dosen, jumlah mata kuliah, jumlah ruangan dan jumlah mahasiswa menjadi tantangan dalam proses penjadwalan karena harus dipertimbangkan agar tidak terjadi bentrok dalam hasil penjadwalan. 
Kebutuhan mahasiswa dalam menyelesaikan masa studinya tidak boleh terkendala hanya karena tidak dapat mengambil mata kuliah yang diwajibkan dikarenakan pelaksanaan perkuliahan yang terbentur dengan waktu pelaksanaan mata kuliah yang lain. 
Selain itu, kebutuhan dosen yang harus meluangkan banyak waktunya untuk melakukan tugas lainnya selain mengajar, juga harus diperhitungkan.

Masalah-masalah yang ada dalam proses penjadwalan mata kuliah ini bisa diminimalisir dengan menggunakan teknologi yang ada sehingga dihasilkan proses penjadwalan yang optimal sesuai dengan batasan-batasan yang ditentukan. 
Ada banyak pendekatan untuk menemukan solusi penjadwalan yang terbaik, yaitu pendekatan Artificial Intelligence (Kecerdasan Buatan), Metaheuristic Methods, Constraint Programming, dan Mathematical Programming. 
Metaheuristic Methods merupakan metode untuk menyelesaikan masalah multivariabel. Salah satu bentuk pendekatan Metaheuristic Method adalah Algoritma Genetika (\cite{ramadhani2021perancangan}). 	
Algoritma Genetika merupakan teknik untuk mencari penyelesaian optimal dari sebuah permasalahan yang memiliki banyak solusi. Teknik ini akan mencari penyelesaian dari beberapa solusi yang ada sampai diperoleh penyelesaian terbaik sesuai dengan kriteria yang telah ditentukan sebelumnya. 
Kriteria-kriteria ini biasa dikenal dengan fitness(\cite{binusAlgoritmaGenetika}).

Berdasarkan latar belakang di atas, pada Tugas Akhir ini diajukan rancangan sistem pemanfaatan algoritma genetika dalam proses penjadwalan mata kuliah di Departemen Teknik Komputer ITS

\section{Rumusan Masalah}

% Ubah paragraf berikut sesuai dengan rumusan masalah dari tugas akhir
Berdasarkan latar belakang di atas, maka dapat dirumuskan masalah pada Tugas Akhir ini sebagai berikut:
\begin{enumerate}
    \item Bagaimana cara membuat dan merancang sistem optimasi proses penjadwalan mata kuliah di Departemen Teknik Komputer ITS dengan memanfaatkan algoritma genetika?
\end{enumerate}

\section{Batasan Masalah atau Ruang Lingkup}

Dalam pembuatan Tugas Akhir ini, pembahasan masalah dibatasi beberapa hal berikut:
\begin{enumerate}
    \item Penjadwalan untuk perkuliahan ini akan dibuat hanya disesuaikan dengan kebutuhan Departemen Teknik Komputer, FTEIC, ITS.
    \item Data yang digunakan diperoleh dari Departemen Teknik Komputer, FTEIC, ITS, semester genap Tahun 2023-2024.
\end{enumerate}

\section{Tujuan}

% Ubah paragraf berikut sesuai dengan tujuan penelitian dari tugas akhir
Berdasarkan latar belakang dan rumusan masalah di atas, didapatkan tujuan pada Tugas Akhir ini sebagai berikut:
\begin{enumerate}
    \item Mampu merancang dan membuat sistem untuk optimasi proses penjadwalan mata kuliah di Departemen Teknik Komputer ITS dengan algoritma genetika.
\end{enumerate}

\section{Manfaat}

% Ubah paragraf berikut sesuai dengan tujuan penelitian dari tugas akhir
Adapun manfaat yang diperoleh dari Tugas Akhir ini sebagai berikut:
\begin{enumerate}
    \item Bagi penulis yaitu menambah pengetahuan dan penerapan mengenai pemanfaatan Algoritma Genetika dalam melakukan proses penjadwalan mata kuliah di Departemen Teknik Komputer 
    \item Bagi mahasiswa yaitu menjadi referensi penelitian yang menggunakan Algoritma Genetika.
\end{enumerate}
